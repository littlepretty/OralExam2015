\section{Introduction}
\label{Sec-Intro}
%today, large-scaled networked system analysis is everywhere and critical to our life. 

Researchers conducting analysis of networked computer systems are often concerned with questions of scale. What is the impact of a system if the communication delay is X times longer, the bandwidth is Y times larger, or the processing speed is Z times faster? Various testbeds have been created to explore answers to those questions before the actual deployment. Ideally, testing on an exact copy of the original system preserves the highest fidelity, but is often technically challenging and economically infeasible, especially for large-scale systems. Simulation-based testbeds can significantly improve the scalability and reduce the cost by modeling the real systems. However, the fidelity of modeled systems is always in question due to model abstraction and simplification. For example, large ISPs today prefer to evaluate the influence of planned changes of their internal networks through tests driven by realistic traffic traces rather than complex simulations. Network emulation is extremely useful for such scenarios, by allowing unmodified network applications being executed inside virtual machines (VMs) over controlled networking environments. This way, scalability and fidelity is well balanced as compared with physical or simulation testbeds.

A handful of network emulators have been created based on various types of virtualization technologies. Examples include DieCast\cite{DieCast}, TimeJails\cite{TimeJails}, VENICE\cite{VirtualTimeMachine} and dONE \cite{RelativisticTime}, which are built using full or para-virtualization (such as Xen\cite{Xen}), as well as Mininet\cite{LaptopSDN, ReproNetExprCBE}, CORE\cite{CORE} and vEmulab\cite{Emulab}. using OS-level virtualization (such as OpenVZ\cite{OpenVZ}, Linux container\cite{LXC} and FreeBSD jails\cite{FreeBSDJails}). All those network emulators offer functional fidelity through the direct execution of unmodified code. Xen enables virtualization of different operating systems, whereas lightweight Linux container enables virtualization at the application level with two orders of magnitude more of VM (or container) instances, i.e., emulated nodes, on a single physical host. In this work, we focus on improving the Linux container technology for scalable network emulation, in particular with the application of software-defined networks (SDN). Mininet\cite{LaptopSDN} is by far the most popular network emulator used by the SDN community\cite{Frenetic, AbsNetUpd, LivMigEntNet}. The linux-container-based design enables Mininet users to experiment ``a network in a laptop" with thousands of emulated nodes. However, Mininet cannot guarantee fidelity at high loads, in particular when the number of concurrent active events is more than the number of parallel cores. For example, on a commodity machine with 2.98 GHz CPU and 4 GB RAM providing 3 Gb/s internal bandwidth, Mininet is only capable to emulate a network up to 30 hosts, each with a 100 MHz CPU and 100 MB RAM and connected by 100 Mb/s links\cite{ReproNetExprCBE}. Emulators cannot reproduce correct behaviors of a real network with large topology and high traffic load because of the limited physical resources. In fact, the same issue occurs in many other VM-based network emulators, because a host \emph{serializes} the execution of multiple VMs, rather than in parallel like a physical testbed. VMs take its notion of time from the host system's clock, and hence time-stamped events generated by the VMs are multiplexed to reflect the host's serialization.

Our approach is to develop the notion of virtual time inside containers to improve fidelity and scalability of the container-based network emulation. A key insight is to trade time for system resources by precisely scaling the system's capacity to match behaviors of the target network. The idea of virtual time has been explored in the form of time-dilation-based\cite{ToInfinityBeyond} and VM-scheduling-based\cite{VirtTimeOpenVZ, SliceTime}  designs, and has been applied to various virtualization platforms including Xen\cite{DieCast}, OpenVZ\cite{VirtTimeOpenVZ}, and Linux Container\cite{TimeKeeper}. In this work, we take a time-dilation-based approach to build a lightweight virtual time system in Linux container, and have integrated the system to Mininet for scalability and fidelity enhancement. The time dilation factor (TDF) is defined as the ratio between the rate at which wall-clock time has passed to the emulated host's perception of time\cite{ToInfinityBeyond}. A TDF of 10 means that for every ten seconds of real time, applications running in a time-dilated emulated host perceive the time advancement as one second. This way, a 100 Mbps link is scaled to a 1 Gbps link from the emulated host's viewpoint.

Our contributions are summarized as follows. First, we have developed an independent and lightweight middleware in the Linux kernel to support virtual time for Linux container. Our system transparently provides the virtual time to processes inside the containers, while returns the ordinary system time to other processes. %The implementation is based on a recent Linux kernel with no dependency of additional libraries.
No change is required in applications, and the integration with network emulators is easy (only slight changes in the initialization routine). Second, to the best of our knowledge, we are the first to apply virtual time in the context of SDN emulation, and have built a prototype system in Mininet. Experimental results indicate that with virtual time, Mininet is capable to precisely emulate much larger networks with high loads, approximately increased by a factor of TDF. Third, we have designed an adaptive time dilation scheme to optimize the performance tradeoff between speed and fidelity. Finally, we have demonstrated the fidelity improvement through a realistic case study about evaluation of the limitations of the equal-cost multi-path (ECMP) routing in data center networks.

The remainder of the paper is structured as follows. Section \ref{Sec-Architecture} presents the virtual time system architecture design. Section \ref{Sec-Implementation} illustrates the implementation of the system and its integration with Mininet. Section \ref{Sec-Experiments} evaluates the virtual-time-enabled Mininet, with a case study of ECMP routing evaluation in Section \ref{Sec-CaseStudy}. Section \ref{Sec-RelatedWorks} discuss existing works regarding to virtual time. Section \ref{Sec-Conclusion} concludes the paper with future works.

%One challenge to addressing such questions is the cost or availability of emerging hardware technologies.  
% because Linux scheduler controls CPU resource
% Modern networking system is usually of great scale and complexity, which make pre-deployment test and evaluation both a must and a headache. In general, \textit{simulation, testbed} and \textit{emulation} are possible surrogates that wield a double-edged sword. Simulators\cite{NS-3,NS-2,CORE} are pure software built upon the mathematical models of network devices, protocols and traffics. As a result, usually they are easy to run and their results can be well reproduced. On the other side, also because of the modeling, one may worry about the fidelity. Testbeds\cite{Emulab, PlanetLab, VINI}, in contrast, are setup with physical hardware. Ideally, they mimic the entire networking system with reduced scale and their result could best capture the characteristics of the target. In practice, unfortunately, testbeds are shared among different researchers or tailored for specific research project\cite{DCTCP, Hedera}. Often they lack the flexibility to conduct custom experiments. Emulators therefore appear as the good-man-in-the-middle. They support user-defined topologies; they just need economic virtual hardware; they are almost as flexible and configurable as simulators; they run real binary code like OS kernel and network applications. 

% The key technique that make emulator a competitive candidate is virtualization. Emulators running full-system emulation \cite{DieCast,TimeJails,ToInfinityBeyond, VirtTimeOpenVZ}often adopt paravirtualiztion, for example Xen\cite{Xen} and OpenVZ\cite{OpenVZ}, to run one VM per host. Such a heavy weight virtualization bring the scalability problem. Then enters the OS-level virtualization. In this type of virtualization, multiple separated user space instances that plays as virtual nodes in emulation can share the kernel of one single operating system. Emulators like Mininet\cite{LaptopSDN, ReproNetExprCBE} CORE\cite{CORE} and others \cite{TimeKeeper, NtwkEmultAdaptVirtTime} that adopts Containers\cite{LXC} or Jails\cite{FreeBSDJails} are able to run experiment with up to tens or hundreds of virtual nodes on a single PC or multicore server. 

% As the best known example, Mininet provides user a flexible, deployable, interactive, scalable, realistic and share-able workflow with lightweight container-based virtualization. The most persuasive proof of Mininet's success is its wide usage by SDN researchers or communities\cite{Frenetic, AbsNetUpd, LivMigEntNet}. With Mininet as network environment, OpenFlow\cite{Openflow} as network protocol, POX or NOX\cite{Nox} as OpenFlow controllers, anyone with a PC can implement a custom network feature or entire new network architecture, test it on large topologies with non-trivial application traffic, and then deploy the exact same code and scripts into real production network\cite{LaptopSDN}.

% \section{Motivation}
% Like many other network emulators, however, Mininet also suffers many limitations. To name a few, it may not guarantee performance fidelity at high loads because Linux scheduler controls CPU resource; it cannot handle different OS kernels simultaneously because of its partial virtualization approach. Among many shortages, the most significant one for Mininet is its experimental scope\cite{ReproNetExprCBE}: Mininet-Hifi targets experiments that have aggregate resource \textit{requirements that fit within} a single modern multi-core server. For example, on a server with 2.98GHz of CPU and 4GB RAM that provides 3Gb/s of internal packet bandwidth, the largest network environment user can build is a topology of nearly 30 host with 100MHz CPU and 100MB RAM each, connected by 100Mb/s links. If we need several 1 Gb/s links, experiment result is doomed, e.g. Mininet cannot guarantee a high-fidelity emulation. Similarly, Mininet is unable to emulate scenarios like computational grids which interconnected by high-speed (usually 10Gpbs) and low latency (200ms round trip time) links.

% With the magic wand of virtual clock, fortunately, we can trade time for emulation resource. Let us see how virtual time magically turns 100Mpbs links in the last example into 1Gpbs links. Suppose an arbitrary host A send 1Gb data to host B, with experiment configuration listed above, the transfer will take 10 seconds. However, if we let B run ten times slow as physical clock, B would thought that the link bandwidth is 1Gb/s because it receives 1Gb data from only within 1 second. It also work for A if A use this virtual clock instead of physical clock. Running the emulation under a clock ten times slower than real time, all hosts will believe that the links are 1Gpbs. Note that at the same time, all host now have 10 times computation capacity than before, e.g. 1GHz CPU for each host. In a word, carefully managing the virtual time in different containers can make Mininet run emulations that not necessarily fit within the physical resources.

% This paper is interested in pushing the scalability of container-based network emulators, like Mininet, even further. First we abstract the architecture of container-based network emulator. Then we propose a generic methodology that integrating virtual time to container-based emulation. Specifically, to let emulators run large scale emulation experiment on resource limited platform, we adopt virtual time in the context of Linux namespace\cite{LinuxNamespace}. One benefit of such implementation is that the advancement of concerned processes in network emulation can be uniformly and accurately slowed down while leaving the others in the same operating system being unaffected. Moreover, to tackle the biggest limitation of virtual time's, we propose an adaptive system to dynamically scheduling the configuration of time dilation so that emulation can be finished with guaranteed fidelity as soon as possible. There are 3 major distinctions separating our work from previous ones. First, our virtual time implementation is on the pure basis of recent Linux kernel; it depends on no additional library or software. Second, we apply virtual time feature provided by our modified Linux kernel to the most widely used SDN network emulator Mininet-Hifi. Third, we develop and plug two additional modules into Mininet to help it dynamically manage virtual time.

% The remainder of this paper is structured as follows. First, we will discuss virtual time in more detail and introduce related works in section \ref{Sec-RelatedWorks}. Then, in section \ref{Sec-Architecture}, we propose the 3-layered architecture of container-based network emulation system and illustrate the generic ideas that extend such architecture with virtual time support. Given Mininet as an exemplar emulator, a concrete implementation is unfolded in section \ref{Sec-Implementation}. To evaluate different aspect of virtual time supported Mininet, we conduct 3 emulation experiments with scale of several hundred nodes. Experiment results and discussions are given in section \ref{Sec-Experiments}. Section \ref{Sec-Conclusions} finally concludes our work.
