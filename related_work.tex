\section{Related Work}
\label{Sec-RelatedWorks}

\subsection{Virtual Time System}
Virtual time has been investigated to improve the scalability and fidelity of virtual-machine-based network emulation. There are two main approaches to develop virtual time systems. The first approach is based on time dilation, a technique to uniformly scale the virtual machine's perception of time by a specified factor. It was first introduced by Gupta et al. \cite{ToInfinityBeyond}, and has been adopted to various types of virtualization techniques and integrated with a handful of network emulators. Examples include DieCast\cite{DieCast}, SVEET\cite{SVEET}, NETbalance\cite{NETbalance}, TimeJails\cite{ComparisonVR-VM, TimeJails} and TimeKeeper\cite{TimeKeeper}. The second approach focuses on synchronized virtual time by modifying the hypervisor scheduling mechanism. Hybrid testing systems that integrate network emulation and simulation have adopted this approach. For example, our previous work\cite{jin2012virtual} integrates an OpenVZ-based virtual time system \cite{VirtTimeOpenVZ} with a parallel discrete-event network simulator by virtual timer.
SliceTime\cite{SliceTime} integrates ns-3\cite{NS-3} with Xen to build a scalable and accurate network testbed.

Our approach is technically closest to TimeKeeper \cite{TimeKeeper} through direct kernel modifications of time-related system calls. The differences are (1) we are the first to apply virtual time in the context of SDN emulation, (2) our system has a wider coverage of system calls interacting in virtual time, and (3) our system has an adaptive time dilation scheduler to balance speed and fidelity for emulation experiments.

\subsection{Adaptive Virtual Time Scheduling}
The key insight of virtual time is to trade time with system resources. Therefore, a primary drawback is the proportionally increased execution time. To determine an appropriate TDF, VENICE \cite{VirtualTimeMachine} proposes a static management scheme to forecast the recourse demand. One problem of static time dilation management is that we often assume the maximum load to ensure fidelity and thus overestimate the scaling factor. 

TimeJails \cite{TimeJails} presents a dynamic management scheme \cite{NtwkEmultAdaptVirtTime} to adjust TDF in run-time based on CPU utilization. 
We take a similar approach with two differences: the target platform and communication overhead. TimeJails is a Xen-based platform extended to a 64-node cluster for scalability, while our system supports more scalable experiments on a single machine with Linux container. TimeJails requires a special protocol to prioritizing TDF request message in local area networks, while the communication overhead of our system is much lower, either through synchronized queues or method invocations among extended modules in the emulator.

% \newpage
\subsection{SDN Emulation and Simulation}
OpenFlow \cite{Openflow} is the first standard communications interface defined between the control and forwarding planes of an SDN architecture. Examples of OpenFlow-based SDN emulation testbeds include Mininet \cite{LaptopSDN}, Mininet-HiFi \cite{ReproNetExprCBE}, EstiNet \cite{EstiNet}, ns-3 \cite{NS-3} and S3FNet\cite{jin2013parallel}.
Mininet is currently the most popular SDN emulator, which uses process-based virtualization technique to provide a lightweight and inexpensive testbed. Ns-3 \cite{NS-3} has an OpenFlow simulation model and also offers a realistic OpenFlow environment through its generic emulation capability, which has been linked with Mininet \cite{MininetLinkNS3}. S3FNet\cite{jin2013parallel} supports scalable simulation/emulation of OpenFlow-based SDN through a hybrid platform that integrates a parallel network simulator and a virtual-time-enabled OpenVZ-based network emulator \cite{S3F_website}. %Mininet support linking simulation with ns-3 \cite{MininetLinkNS3}.


% \subsection{Virtual Time}

% Nowadays, it is common to see emulators taking the advantage of virtualization technology like OpenVZ\cite{OpenVZ} or Xen\cite{Xen} to emulate large scale distributed system on single server or moderate cluster. After all, virtual machines now are easy to setup and control and have been always cost-efficient. Virtualization based emulator is also of high functional fidelity because in emulations real code are executed. However, it is limited by its limited physical resource when the scale of targeted distributed or network system increases. Hence, virtual time is widely used by many researchers to improve the scalability and efficiency of their emulators or testbeds.

% For example, Zheng and Nicol\cite{VirtTimeOpenVZ} have modified OpenVZ and OpenVZ's schedulers so as to measure the time used by virtual machine in computation and have Linux return virtual time to virtual machines. Lamps and Nicol\cite{TimeKeeper} design and implement a lightweight virtual time system TimeKeeper and integrated it with CORE\cite{CORE}. Our work bears similarity to that of the TimeKeeper, all take the advantage of Linux container mechanism. However, we want to use a simpler method to inject virtual time to Mininet's virtual hosts. 

% Many researchers\cite{ToInfinityBeyond, ComparisonVR-VM, SVEET, NETbalance} implemented virtual time on Xen virtual machines. Aside from implementation details, generally speaking, \textit{two} major modifications are needed to support time dilation in their emulation system. The first one goes to Xen's hypervisor (VM Monitor). It includes a shared data structure of wall clock time passing between Xen VMM and guesting OS and timer interrupts. Second, developers also need to modify guest OS's kernel to read appropriately scaled version of the hardware Time Stamp Counter (TSC). 

% \subsection{Adaptive Scheduling}

% The magic of virtual time, of course, comes with a cost. All the works mentioned above has two major problems. The first one is that it is subtle to select an adequate dilation value for entire experiment duration. A simple solution used by many emulators\cite{ToInfinityBeyond, SVEET, TimeJails} is to run the entire experiment with a enough large time dilation factor so that dilated resource matches that of the target system's. Sometimes this will lead to unacceptable running time. The second problem is that load generated by the experiment scenario may vary over time. Without a dynamic management scheme, one can only select  for the period with maximum load. When the experiment goes into a low load phase, emulator with fixed  cannot intelligently speed up the progress of the experiment.

% To solve the first problem, VENICE\cite{VirtualTimeMachine} has been adopted a two-level management scheme. At global level, they formulate the problem of finding appropriate time dilation factor as a optimal graph partition problem subjecting to resource constrains. At local level, they proposed load-driven and deadline-driven resource management solutions on VM scheduler, both depending on proper predictive method to forecasting resource needed with respect to future workload or progress deadline. Since time dilation factor can be determined when user setups emulation, we refer it as \textit{static virtual time management solution.}

% \cite{TimeJails} introduced the concept of Epochs to schedule time dilation factor. During an epoch, TDF remains constant and all physical nodes run under same TDF. At an epoch transition, all VMs synchronously switch to a new TDF. In an Epoch, a monitor collect information of emulation system load such as percentage of CPU time used by VMs and utilization of physical network. Whenever the emulator exceeds an overload or falls behind a underload warning threshold, a coordinator should initiate epoch transition and increase or decrease TDF. In their further works\cite{NtwkEmultAdaptVirtTime, NETapproach}, they implemented a simple but effective TDF adaptation scheme on the basis of the threshold of system load. We refer their work as \textit{dynamic virtual time management solution.}
